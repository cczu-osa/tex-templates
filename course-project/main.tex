\documentclass[a4paper, 11pt]{ctexart}

\usepackage{xfrac}
\usepackage{geometry}
\usepackage{fancyhdr}
\usepackage{lastpage}
\usepackage{minted}
\usepackage{fontspec}
\usepackage{graphicx}
\usepackage{adjustbox}
\usepackage{tabularx}
\usepackage{multirow}
\usepackage{dblfloatfix}
\usepackage{hyperref}
\usepackage{wrapfig}
\usepackage{float}
\usepackage{caption}
\usepackage{booktabs}
\usepackage{chngcntr}
\usepackage{amsmath}
\usepackage{amssymb}
\usepackage{ntheorem}
\usepackage[nottoc]{tocbibind}

% 设置页面
\geometry{hmargin=3.18cm, vmargin=2.54cm} % 页边距

% 添加空白页命令
\newcommand{\blankpage}{\phantom{s}\thispagestyle{empty}}

% 设置字体
\setmainfont{Noto Serif} % 如果没有安装 Noto 字体,可以把这三行注释掉,ctex 会使用默认 CJK 字体
\setsansfont{Noto Sans}
\setCJKsansfont{Noto Sans CJK SC}
\setmonofont[Scale=0.95]{Consolas}
\setCJKmonofont[Scale=0.95]{Microsoft YaHei}

% 设置代码默认显示效果
\setminted{autogobble, baselinestretch=1, breaklines, frame=lines, framesep=3mm, style=vs, stripnl, tabsize=4}
\floatname{listing}{代码清单}
\newcommand{\codeinline}{\mintinline{text}}

% 修正代码清单的标题上边距
\let\oldcaption\caption
\makeatletter
\renewcommand{\caption}[1]{
    \ifnum\strcmp{\@currenvir}{listing}=0
    \vspace{-0.3cm}\oldcaption{#1}
    \else
    \oldcaption{#1}
    \fi
}
\makeatother

% 设置表格
\newcolumntype{C}{>{\centering\arraybackslash}X}

% 设置标号
\counterwithin{figure}{section}
\counterwithin{table}{section}
\counterwithin{listing}{section}
\counterwithin{equation}{section}

% 设置定理、例题等特殊块
\theoremstyle{plain}
\theoremheaderfont{\kern+2em\normalfont\bfseries} % 首行缩进 2 格
\theorembodyfont{\normalfont\bfseries} % 定义、定理使用粗体
\newtheorem{definition}{定义}[section]
\newtheorem{theorem}{定理}[section]
\newtheorem{corollary}{推论}[theorem]
\newtheorem{lemma}[theorem]{引理}
\theorembodyfont{\normalfont} % 其它使用正常字体
\newtheorem*{proof}{证明}
\newtheorem{example}{例}[section]
\newtheorem*{solution}{解}
\newtheorem*{remark}{注}


% 设置页面
\pagestyle{fancy}
\fancyhf{}
\chead{常州大学课程设计} % 页眉
\renewcommand{\headwidth}{\textwidth} % 页眉分割横线
\cfoot{第 \thepage 页,共 \pageref{LastPage} 页} % 页脚,显示页码

% 添加快速引用代码文件指令
\newmintedfile[cppfile]{cpp}{}
\newmintedfile[javafile]{java}{}

% 设置图片路径
\graphicspath{ {./images/} {./figures/} }

\begin{document}

% 封面
\begin{titlepage}

    \begin{center}
        \marginbox{0cm 0cm 0cm 3cm}{
            \includegraphics[width=2.67cm, height=2.67cm]{logo.png}
            \marginbox{2cm 0cm 0cm 0cm}{\includegraphics[height=2.67cm]{logo2}}
        }
        \marginbox{0cm 0cm 0cm 3cm}{
            \fontsize{42pt}{0}\selectfont
            课 \enspace 程 \enspace 设 \enspace 计
        }
    \end{center}

    \begin{table}[!b]
        \fontsize{14pt}{0}\selectfont
        \begin{tabularx}{\textwidth}{lXlX}
            \rule{0pt}{36pt}
            课程名称     & \multicolumn{3}{l}{}                        \\
            \cline{2-4}
            \rule{0pt}{36pt}
            题  目     & \multicolumn{3}{l}{}                        \\
            \cline{2-4}
            \rule{0pt}{36pt}
            学生姓名     &                      & \quad 学  号     & \\
            \cline{2-2} \cline{4-4}
            \rule{0pt}{36pt}
            学  院     &                      & \quad 专业班级     & \\
            \cline{2-2} \cline{4-4}
            \rule{0pt}{36pt}
            校内指导老师 &                      & \quad 专业技术职务 & \\
            \cline{2-2} \cline{4-4}
            \rule{0pt}{36pt}
            校外指导老师 &                      & \quad 专业技术职务 & \\
            \cline{2-2} \cline{4-4}
            \rule{0pt}{36pt}
        \end{tabularx}
    \end{table}

\end{titlepage}

\clearpage
\blankpage % 空白页,方便双面打印

% 目录页
\clearpage
\tableofcontents
\thispagestyle{empty} % 隐藏目录页的页码

\clearpage
\blankpage % 空白页,方便双面打印

\clearpage
\setcounter{page}{1} % 从正文第 1 页开始计算页码

\section{通用}

\subsection{图片}

常州大学的 logo 如图 \ref{fig:cczu},这里使用了 \codeinline{\ref{fig:cczu}} 来引用图片标号。

\begin{figure}[htp]
    \centering\includegraphics[width=0.5\textwidth]{cczu}
    \caption{常州大学 logo}
    \label{fig:cczu}
\end{figure}

\subsection{表格}

表 \ref{tab:student-table} 是一个学生信息表。

\begin{table}[htp]
    \centering
    \begin{tabularx}{\textwidth}{XXX}
        \toprule  %添加表格头部粗线
        姓名 & 学号 & 性别 \\
        \midrule  %添加表格中横线
        小明 & 001  & 男   \\
        小红 & 002  & 女   \\
        \bottomrule %添加表格底部粗线
    \end{tabularx}
    \caption{学生信息}
    \label{tab:student-table}
\end{table}

\subsection{列表}

有序列表:

\begin{enumerate}
    \item 第一项
    \item 第二项
    \item 第三项
\end{enumerate}

无序列表:

\begin{itemize}
    \item 第一项
    \item 第二项
    \item 第三项
\end{itemize}

\subsection{引用参考文献}

Ths document is an example of BibTeX using in bibliography management. Three items
are cited: \textit{The \LaTeX\ Companion} book \cite{latexcompanion}, the Einstein
journal paper \cite{einstein}, and the Donald Knuth's website \cite{knuthwebsite}.
The \LaTeX\ related items are \cite{latexcompanion,knuthwebsite}.

这里是一个中文文献的测试 \cite{wikipedia-cn}。

\clearpage
\section{编程相关}

\codeinline{some_function('参数')} 是一个行内代码,下面是一个片段代码:

\begin{minted}{python}
    def main():
        print('Hello, world!')
\end{minted}

代码清单 \ref{snippet:cpp-hello} 是一个有标号的代码清单。

\begin{listing}[htp]
    \inputminted{cpp}{snippets/main.cpp}
    \caption{C++ 的 Hello World 小程序}
    \label{snippet:cpp-hello}
\end{listing}

如果通篇经常使用同一编程语言,可以通过 \codeinline{\javafile{Filename.java}} 快速引入,见 \TeX 源码第 41 和 42 行的 \codeinline{newmintedfile} 指令。引入效果见代码清单 \ref{snippet:java-hello}。

\begin{listing}[htp]
    \inputminted{cpp}{snippets/Hello.java}
    \caption{Java 的 Hello World 小程序}
    \label{snippet:java-hello}
\end{listing}

\clearpage
\section{数学相关}

\subsection{基本}

数学相关的文章通常使用英文标点, \LaTeX 能正确处理中英文字和标点混杂的情况, 比如本 section 使用的是英文标点.

下面是个带标号的数学公式:

\begin{equation}
    \Gamma(\alpha) = \int_{0}^{+\infty} x^{\alpha - 1} e^{-x} \,dx (\alpha > 0)
    \label{eq:gamma-fn}
\end{equation}

公式 \ref{eq:gamma-fn} 是 $\Gamma$ 函数.

这一个行内公式: $f(x) = ax + b$, 以及另一个: $\int \cos x \,dx = \sin x$. 下面是没有标号的块公式:

$$
    E = mc^2
$$

\subsection{定理, 推论, 例题}

\begin{theorem}
    这里是一个很长很长很长很长很长很长很长很长很长很长很长很长很长很长很长很长很长很长很长很长很长很长定理.
    \label{th:1}
\end{theorem}

定理 \ref{th:1} 同样可以通过 \codeinline{\ref} 引用标号.

\begin{proof}
    这里是定理 \ref{th:1} 的证明.
\end{proof}

\begin{example}
    这是一个例题. 包含一个公式:

    \begin{equation}
        \Gamma(\alpha) = \int_{0}^{+\infty} x^{\alpha - 1} e^{-x} \,dx (\alpha > 0)
    \end{equation}

    \label{eg:1}
\end{example}

\begin{solution}
    这是例 \ref{eg:1} 的解.
\end{solution}

\begin{remark}
    这是一个注.
\end{remark}

\clearpage

\bibliographystyle{unsrt}
\bibliography{main}

\end{document}
