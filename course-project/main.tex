\documentclass[a4paper, 11pt]{article}

\usepackage[UTF8]{ctex}
\usepackage{amsmath}
\usepackage{xfrac}
\usepackage{geometry}
\usepackage{fancyhdr}
\usepackage{lastpage}
\usepackage{minted}
\usepackage{fontspec}
\usepackage{graphicx}
\usepackage{adjustbox}
\usepackage{tabularx}
\usepackage{multirow}
\usepackage{dblfloatfix}
\usepackage{hyperref}
\usepackage{wrapfig}
\usepackage{float}
\usepackage{caption}

% 设置页面
\geometry{hmargin=3.18cm, vmargin=2.54cm} % 页边距
\pagestyle{fancy}
\fancyhf{}
\chead{常州大学课程设计} % 页眉
\renewcommand{\headwidth}{\textwidth} % 页眉分割横线
\cfoot{第 \thepage 页,共 \pageref{LastPage} 页} % 页脚,显示页码

% 设置字体
\setmainfont{Noto Serif}
\setsansfont{Noto Sans}
\setCJKsansfont{Noto Sans CJK SC}
\setmonofont[Scale=0.95]{Noto Sans Mono}
\setCJKmonofont[Scale=0.95]{Noto Sans Mono CJK SC}

% 设置代码默认显示效果
\setminted{autogobble, baselinestretch=1, breaklines, frame=lines, framesep=4mm, style=vs, stripnl, tabsize=4}
\newfloat{snippet}{p}{}
\floatname{snippet}{代码片段}
\newcommand{\codeinline}{\mintinline{text}}
\newmintedfile[cppfile]{cpp}{}
\newmintedfile[javafile]{java}{}

% 设置图片目录
\graphicspath{ {./images/} {./figures/} }

% 设置表格
\newcolumntype{C}{>{\centering\arraybackslash}X}

\begin{document}

% 封面
\begin{titlepage}

    \begin{center}
        \marginbox{0cm 0cm 0cm 3cm}{
            \includegraphics[width=2.67cm, height=2.67cm]{logo.png}
            \marginbox{2cm 0cm 0cm 0cm}{\includegraphics[height=2.67cm]{logo2}}
        }
        \marginbox{0cm 0cm 0cm 3cm}{
            \fontsize{42pt}{0}\selectfont
            课 \enspace 程 \enspace 设 \enspace 计
        }
    \end{center}

    \begin{table}[!b]
        \fontsize{14pt}{0}\selectfont
        \begin{tabularx}{\textwidth}{lXlX}
            \rule{0pt}{36pt}
            课程名称 & \multicolumn{3}{l}{} \\
            \cline{2-4}
            \rule{0pt}{36pt}
            题  目 & \multicolumn{3}{l}{} \\
            \cline{2-4}
            \rule{0pt}{36pt}
            学生姓名 & & \quad 学  号 & \\
            \cline{2-2} \cline{4-4}
            \rule{0pt}{36pt}
            学  院 & & \quad 专业班级 & \\
            \cline{2-2} \cline{4-4}
            \rule{0pt}{36pt}
            校内指导老师 & & \quad 专业技术职务 & \\
            \cline{2-2} \cline{4-4}
            \rule{0pt}{36pt}
            校外指导老师 & & \quad 专业技术职务 & \\
            \cline{2-2} \cline{4-4}
            \rule{0pt}{36pt}
        \end{tabularx}
    \end{table}

\end{titlepage}

% 目录页
\tableofcontents

\clearpage
\section{引言}

\subsection{开发背景}

\codeinline{some_function()} 是一个行内代码,下面是一个片段代码:

\begin{minted}{python}
    def main():
        print('Hello, world!')
\end{minted}

常州大学的 logo 如图 \ref{fig:cczu}。

\begin{figure}[htp]
	\centering\includegraphics[width=0.9\textwidth]{cczu}
	\caption{常州大学 logo}
	\label{fig:cczu}
\end{figure}

代码片段 \ref{snippet:cpp-hello} 是一个有标号的代码片段。

\begin{snippet}
	\inputminted{cpp}{snippets/main.cpp}
	\caption{C++ 的 Hello World 小程序}
	\label{snippet:cpp-hello}
\end{snippet}

如果通篇经常使用同一编程语言,可以通过 \codeinline{\javafile{Filename.java}} 快速引入,见 TeX 源码第 41 和 42 行的 \codeinline{newmintedfile} 指令。引入效果见代码片段 \ref{snippet:java-hello}。

\begin{snippet}
	\inputminted{cpp}{snippets/Hello.java}
	\caption{Java 的 Hello World 小程序}
	\label{snippet:java-hello}
\end{snippet}

\clearpage
\section{需求分析}

\clearpage
\section{涉及的关键技术}

\clearpage
\section{系统设计}

\clearpage
\section{系统实现}

\clearpage
\section{结论}

\clearpage
\section{参考资料}

\begin{enumerate}
    \item 第 1 项
    \item 第 2 项
\end{enumerate}

\end{document}
