\documentclass[a4paper, 11pt]{ctexart}

\usepackage{amsmath}
\usepackage{xfrac}
\usepackage{geometry}
\usepackage{fancyhdr}
\usepackage{lastpage}
\usepackage{minted}
\usepackage{fontspec}
\usepackage{graphicx}
\usepackage{adjustbox}
\usepackage{tabularx}
\usepackage{multirow}
\usepackage{dblfloatfix}
\usepackage{hyperref}
\usepackage{wrapfig}
\usepackage{float}
\usepackage{caption}
\usepackage{booktabs}
\usepackage{chngcntr}
\usepackage{mathptmx}

% 设置页面
\geometry{hmargin=3.18cm, vmargin=2.54cm} % 页边距
\pagestyle{fancy}
\fancyhf{}
\chead{常州大学课程设计} % 页眉
\renewcommand{\headwidth}{\textwidth} % 页眉分割横线
\cfoot{第 \thepage 页,共 \pageref{LastPage} 页} % 页脚,显示页码

% 设置字体
\setmainfont{Noto Serif} % 如果没有安装 Noto 字体,可以把这三行注释掉,ctex 会使用默认 CJK 字体
\setsansfont{Noto Sans}
\setCJKsansfont{Noto Sans CJK SC}
\setmonofont[Scale=0.95]{Consolas}
\setCJKmonofont[Scale=0.95]{Microsoft YaHei}

% 设置代码默认显示效果
\setminted{autogobble, baselinestretch=1, breaklines, frame=lines, framesep=3mm, style=vs, stripnl, tabsize=4}
\floatname{listing}{代码清单}
\newcommand{\codeinline}{\mintinline{text}}
\newmintedfile[cppfile]{cpp}{}
\newmintedfile[javafile]{java}{}

% 设置图片
\graphicspath{ {./images/} {./figures/} }

% 设置表格
\newcolumntype{C}{>{\centering\arraybackslash}X}

% 设置标号
\counterwithin{figure}{section}
\counterwithin{table}{section}
\counterwithin{listing}{section}
\counterwithin{equation}{section}

% 修正代码清单的标题上边距
\let\oldcaption\caption
\makeatletter
\renewcommand{\caption}[1]{
    \ifnum\strcmp{\@currenvir}{listing}=0
    \vspace{-0.3cm}\oldcaption{#1}
    \else
    \oldcaption{#1}
    \fi
}
\makeatother

\begin{document}

% 封面
\begin{titlepage}

    \begin{center}
        \marginbox{0cm 0cm 0cm 3cm}{
            \includegraphics[width=2.67cm, height=2.67cm]{logo.png}
            \marginbox{2cm 0cm 0cm 0cm}{\includegraphics[height=2.67cm]{logo2}}
        }
        \marginbox{0cm 0cm 0cm 3cm}{
            \fontsize{42pt}{0}\selectfont
            课 \enspace 程 \enspace 设 \enspace 计
        }
    \end{center}

    \begin{table}[!b]
        \fontsize{14pt}{0}\selectfont
        \begin{tabularx}{\textwidth}{lXlX}
            \rule{0pt}{36pt}
            课程名称     & \multicolumn{3}{l}{}                        \\
            \cline{2-4}
            \rule{0pt}{36pt}
            题  目     & \multicolumn{3}{l}{}                        \\
            \cline{2-4}
            \rule{0pt}{36pt}
            学生姓名     &                      & \quad 学  号     & \\
            \cline{2-2} \cline{4-4}
            \rule{0pt}{36pt}
            学  院     &                      & \quad 专业班级     & \\
            \cline{2-2} \cline{4-4}
            \rule{0pt}{36pt}
            校内指导老师 &                      & \quad 专业技术职务 & \\
            \cline{2-2} \cline{4-4}
            \rule{0pt}{36pt}
            校外指导老师 &                      & \quad 专业技术职务 & \\
            \cline{2-2} \cline{4-4}
            \rule{0pt}{36pt}
        \end{tabularx}
    \end{table}

\end{titlepage}

% 空白页,方便双面打印
\clearpage
\phantom{s}
\thispagestyle{empty}

% 目录页
\clearpage
\tableofcontents
\thispagestyle{empty} % 隐藏目录页的页码

% 空白页,方便双面打印
\clearpage
\phantom{s}
\thispagestyle{empty}

\clearpage
\setcounter{page}{1} % 从正文第 1 页开始计算页码
\section{引言}

\subsection{开发背景}

\codeinline{some_function('参数')} 是一个行内代码,下面是一个片段代码:

\begin{minted}{python}
    def main():
        print('Hello, world!')
\end{minted}

常州大学的 logo 如图 \ref{fig:cczu},这里使用了 \codeinline{\ref{fig:cczu}} 来引用图片标号。

\begin{figure}[htp]
    \centering\includegraphics[width=0.5\textwidth]{cczu}
    \caption{常州大学 logo}
    \label{fig:cczu}
\end{figure}

代码清单 \ref{snippet:cpp-hello} 是一个有标号的代码清单。

\begin{listing}[htp]
    \inputminted{cpp}{snippets/main.cpp}
    \caption{C++ 的 Hello World 小程序}
    \label{snippet:cpp-hello}
\end{listing}

如果通篇经常使用同一编程语言,可以通过 \codeinline{\javafile{Filename.java}} 快速引入,见 \TeX 源码第 41 和 42 行的 \codeinline{newmintedfile} 指令。引入效果见代码清单 \ref{snippet:java-hello}。

\begin{listing}[htp]
    \inputminted{cpp}{snippets/Hello.java}
    \caption{Java 的 Hello World 小程序}
    \label{snippet:java-hello}
\end{listing}

\subsection{目标}

表 \ref{tab:student-table} 是一个学生信息表。

\begin{table}[htp]
    \centering
    \begin{tabularx}{\textwidth}{XXX}
        \toprule  %添加表格头部粗线
        姓名 & 学号 & 性别 \\
        \midrule  %添加表格中横线
        小明 & 001  & 男   \\
        小红 & 002  & 女   \\
        \bottomrule %添加表格底部粗线
    \end{tabularx}
    \caption{学生信息}
    \label{tab:student-table}
\end{table}

下面是个数学公式:

\begin{equation}
    \Gamma(\alpha) = \int_{0}^{+\infty} x^{\alpha - 1} e^{-x} \,dx (\alpha > 0)
    \label{eq:gamma-fn}
\end{equation}

公式 \ref{eq:gamma-fn} 是 $\Gamma$ 函数。

再展示一个行内公式 $f(x) = ax + b$。以及没有标号的块公式:

$$
E = mc^2
$$

本段是使用英文标点的例子, 因为在数学语境下常常需要用英文标点. \LaTeX 会正确地在英文标点和汉字之间加一些间距, 就像数学书上那样. 再来测试一个行内公式: $\int \cos x \,dx = \sin x$.

\clearpage
\section{需求分析}

\clearpage
\section{涉及的关键技术}

\clearpage
\section{系统设计}

\clearpage
\section{系统实现}

\clearpage
\section{结论}

\clearpage
\section{参考资料}

\begin{enumerate}
    \item 第 1 项
    \item 第 2 项
\end{enumerate}

\end{document}
