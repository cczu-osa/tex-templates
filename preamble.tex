\usepackage{xfrac}
\usepackage{geometry}
\usepackage{fancyhdr}
\usepackage{lastpage}
\usepackage{minted}
\usepackage{fontspec}
\usepackage{graphicx}
\usepackage{adjustbox}
\usepackage{tabularx}
\usepackage{multirow}
\usepackage{dblfloatfix}
\usepackage{hyperref}
\usepackage{wrapfig}
\usepackage{float}
\usepackage{caption}
\usepackage{booktabs}
\usepackage{chngcntr}
\usepackage{amsmath}
\usepackage{amssymb}
\usepackage{ntheorem}
\usepackage[nottoc]{tocbibind}

% 设置页面
\geometry{hmargin=3.18cm, vmargin=2.54cm} % 页边距

% 添加空白页命令
\newcommand{\blankpage}{\phantom{s}\thispagestyle{empty}}

% 设置字体
\setmainfont{Noto Serif} % 如果没有安装 Noto 字体,可以把这三行注释掉,ctex 会使用默认 CJK 字体
\setsansfont{Noto Sans}
\setCJKsansfont{Noto Sans CJK SC}
\setmonofont[Scale=0.95]{Consolas}
\setCJKmonofont[Scale=0.95]{Microsoft YaHei}

% 设置代码默认显示效果
\setminted{autogobble, baselinestretch=1, breaklines, frame=lines, framesep=3mm, style=vs, stripnl, tabsize=4}
\floatname{listing}{代码清单}
\newcommand{\codeinline}{\mintinline{text}}

% 修正代码清单的标题上边距
\let\oldcaption\caption
\makeatletter
\renewcommand{\caption}[1]{
    \ifnum\strcmp{\@currenvir}{listing}=0
    \vspace{-0.3cm}\oldcaption{#1}
    \else
    \oldcaption{#1}
    \fi
}
\makeatother

% 设置表格
\newcolumntype{C}{>{\centering\arraybackslash}X}

% 设置标号
\counterwithin{figure}{section}
\counterwithin{table}{section}
\counterwithin{listing}{section}
\counterwithin{equation}{section}

% 设置定理、例题等特殊块
\theoremstyle{plain}
\theoremheaderfont{\kern+2em\normalfont\bfseries} % 首行缩进 2 格
\theorembodyfont{\normalfont\bfseries} % 定义、定理使用粗体
\newtheorem{definition}{定义}[section]
\newtheorem{theorem}{定理}[section]
\newtheorem{corollary}{推论}[theorem]
\newtheorem{lemma}[theorem]{引理}
\theorembodyfont{\normalfont} % 其它使用正常字体
\newtheorem*{proof}{证明}
\newtheorem{example}{例}[section]
\newtheorem*{solution}{解}
\newtheorem*{remark}{注}
